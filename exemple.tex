\documentclass[phd]{dissertum}
\usepackage{lipsum}
\usepackage[printonlyreused]{acronym}


% Variables à modifier
% ==========================================================
\titre{Vers une théorie unifiée des sciences}
\soustitre{Étude préliminaire}
\auteur{Issey Tou}
\date{Août}{2023}

\faculte{Faculté des arts et des sciences}
\departement{Département des sciences unifiées}
\programme{unification des sciences, option omniscience}

\president{Big Boss}
\directrice{Méchante Bolle}
\codirecteur{Moyen Génie}
\examinatrice{Star Académique}
\membre{Simple Quidam}
% ==========================================================


\begin{document}

  % Page titre et identification du jury
  \pagetitre
  \pagejury

  % Résumé et mots-clés en français et en anglais
  \begin{resume}{unification, tout}
    Nous cherchons à unifier toutes les sciences et démontrons que c'est compliqué.
  \end{resume}

  \begin{abstract}{unification, everything}
    We try to unify all sciences and prove that it is complicated.
  \end{abstract}

  % Table des matières, liste des figures, liste des tableaux
  \matieres

  % Liste des sigles
  \prechapter{Liste des sigles et abréviations}
    \begin{acronym}
      \acro{UdeM}{Université de Montréal}
      \acro{FAS}{Faculté des arts et des sciences}
      \acro{NPU}{N'est pas utilisé!}
    \end{acronym}

  % Dédicace (facultative)
  \dedicace{À toutes les personnes qui aiment la science}

  % Remerciements (chapitre facultatif)
  \prechapter{Remerciements}
    Merci.

  % Avant-propos (chapitre facultatif)
  \prechapter{Avant-propos}
    \lipsum[1-2]

  % Corps du document
  % ==========================================================

  \chapter{Introduction}
    On parle de la \ac{FAS} de l'\ac{UdeM}.
    \lipsum[3-6]

  \chapter{Méthodologie, résultats et discussion}
    \lipsum[7-8]\cite{knuth84}
    \section{Méthodologie}
      \lipsum[9-10]
    \section{Résultats}
      \lipsum[11-12]
      \subsection{Expérience 1}
        \lipsum[13]
        \begin{figure}
          \centering
          \fbox{joli graphe}
          \caption{Nombre d'articles publiés par domaine}
        \end{figure}
      \subsection{Expérience 2}
        \lipsum[14]
        \begin{figure}
          \centering
          \fbox{joli graphe}
          \caption{Nombre d'articles publiés par année}
        \end{figure}
    \section{Discussion}
      \lipsum[15-19]
      \begin{table}
        \centering
        \begin{tabular}{ll}
          \toprule
          \bf Nom & \bf Citations \\
          \midrule
          Etal & 2\,340\,285 \\
          Aparaître & 270\,859 \\ 
          \bottomrule
        \end{tabular}
        \caption{Les personnes les plus citées}
      \end{table}

  \chapter{Conclusion}
    On parle encore de la \ac{FAS}.
    \lipsum[7-10]

  % ==========================================================
  
  % Bibliographie
  \bibliographystyle{apalike-qc}
  \bibliography{biblio}
  
  % Annexes
  \appendix
  \chapter{Données}
    \lipsum[11-20]

\end{document}